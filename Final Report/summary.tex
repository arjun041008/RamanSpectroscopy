% This is summary.tex

The presence of bacteria in milk varies the chemical composition of a milk sample, which can be detected using spectroscopic methods like Raman spectroscopy. Employing the power of predictive modelling to this use case of differentiating between bacterial and non-bacterial spectra has valuable contributions in quality control in milk samples. This study on using Raman spectroscopy and machine learning for detecting \textit{S. aureus} bacteria in milk not only focuses on producing maximum accuracy in the binary classification task but also aims to attribute individual predictions to specific features in the spectra, thus enhancing interpretability. This pipeline, overall, produces an easily discernible, fast, cost-effective, and scalable method for detecting bacterial contamination in milk. 